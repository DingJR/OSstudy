\documentclass{article}

\usepackage{amsmath}
\usepackage{graphicx}
\usepackage{subcaption}
\usepackage{cite}
\usepackage{siunitx}
\usepackage{multirow}
\usepackage{booktabs}
\usepackage{longtable}
\usepackage{tikz}
\usepackage{listings}
\usepackage{xcolor} 
\usepackage{amsfonts}

\lstset{
    numberstyle= \tiny,     
    keywordstyle= \color{blue!70},
    commentstyle=\color{red!50!green!50!blue!50},    
    frame=shadowbox,    
    rulesepcolor= \color{red!20!green!20!blue!20}  } 

\sisetup{
    round-mode = places,
    round-precision = 2,
}


\begin{document}

\section{README}
	\subsection{Usage}
		\begin{lstlisting}
./mysh
./mysh inputfile 
		\end{lstlisting}
	\subsection{Data Structure}
		\begin{itemize}
			\item JOB
			JOB stores every command line and its relating information.
					\begin{lstlisting}
struct JOB
{
char *command;		
//command Line
char **argv;		
//parameters
char *inputFile;	
char *outputFile;
int argNum;			
//number of parameters
pid\_t jobid;		
//process id
};
					\end{lstlisting}
			\item COMMAND\_EXE
					\begin{lstlisting}
struct COMMAND\_EXE
{
int mode;                   	
//BACKGROUND or FOREGROUND or PIPELINE
struct JOB *task[MAX\_PIPELINE];	
//maybe pipelined, separate it to several parts 
int taskNum ;					
//numofpipeline + 1
pid\_t cmdid;					
//process id
};
					\end{lstlisting}
			\item SHELLINFO
				\begin{lstlisting}
struct SHELLINFO
{
	char wd[PATH\_SIZE];					 
	//current working directory
	int  jobs;                  		 
	//exclude command "wait"
	struct COMMAND\_EXE* tsk[MAX\_COMMAND];
	//all the JOBS
}shell;
				\end{lstlisting}
			\end{itemize}
	\subsection{Function}
			\begin{itemize}
				\item char *ReadLine	\\
					ReadLine read a line or to the end of file, as an extend of fgets().
					For the memory allocation reason(we don't know how big the memory block should malloc beforehand), I malloc and fgets, and realloc and fgets again and again until the space satisfy the request. Also, in some annoying test script which lacks of a endline in the end of file, the function can add a '$\backslash$n' to it.
				\item struct COMMAND\_EXE* Split\_Line\_To\_Segment	\\
					separate the command line by '$\backslash$n'. What's more, deal with '$\backslash$'','$\backslash$"'(Extract the contents from every couple of quotations)
				\item int  Split\_Line	\\
					And I begin to store every option, every argv and command name into argv[][]. For the redirection requirement, the function also points out what input file is and output file is.(if no specification, stdin and stdout as default)
				\item void Csh\_Execute	\\
					Main execution function. If the command is built in, execute it by ourselves. If not, call execvp(). There are some differences in FOREGROUND and BACKGROUND mode. 
					\begin{itemize}
						\item FOREGROUND
						call block waitpid
						\item BACKGROUND
						call non-block waitpid
					\end{itemize}
                \item void Csh\_Exe\_Command\_Recursive(PipeLine Accomplish) \\
					I haven't figured out a way to combine pipeline mode with normal mode(some principles unclear, like the redirection and pipeline's priority), so I write a simple recursive function to realize the pipeline. Here's general method: except the leftmost and rightmost of pipeline, every sub-command's input is from pipe(derived from its parent process )'s read-end, and output is passing through a new pipe. Just paste the core code.
				\begin{lstlisting}
void Csh\_Exe\_Command\_Recursive(int i, 
struct COMMAND\_EXE* cmd,int outfd[2])
{
	int infd[2] ; 
	int io[2];
	io[0] = STDIN\_FILENO;
	io[1] = STDOUT\_FILENO;
	if(i!=0)
		pipe(infd);
	if(i!=0)
	{
		pid\_t child\_id = fork();
		if(child\_id == 0)
		{
		Csh\_Exe\_Command\_Recursive(i-1,cmd,infd);
		}
		else 
		{
		wait(0);
		}
		}
		if(i==0)
		{
		close(outfd[0]);
		dup2(outfd[1],STDOUT\_FILENO);
		io[1] = outfd[1];
		}
		else if(i==(cmd->taskNum-1))
		{
		close(infd[1]);
		dup2(infd[0],STDIN\_FILENO);
		io[0] = infd[0];
		}
		else 
		{
		close(infd[1]);
		close(outfd[0]);
		dup2(outfd[1],STDOUT\_FILENO);
		dup2(infd[0],STDIN\_FILENO);
		io[0] = infd[0];
		io[1] = outfd[1];
		}
}
				\end{lstlisting}
				
				\item Release\_Zombie
					Initially, I thought it should be a skillful task, and I was track off into a relatively more complicated way to solve it. First, I used the share memory to achieve processes' communication. Then I used signal(SIGCHILD,ZombieClr) to awake the ZombieClr()(which function used to release Zombies). What's more, to avoid conflicts between different processes, I used mutex to realize mutual exclusive. But then I found that I zigzagged...After that, I just use waitpid whose option is set to WNOANG to all the processes every period of time. And it does work better than the former version...
			\end{itemize}	
\end{document}

